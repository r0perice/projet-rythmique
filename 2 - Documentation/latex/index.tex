\hypertarget{index_intro_sec}{}\section{Introduction}\label{index_intro_sec}
Projet regroupant Art et Réalité Virtuelle en collaboration avec l’association brestoise « Vivre Le Monde » (Body percussionnistes). Le but du projet est de permettre aux artistes de jouer de la musique en utilisant la reconnaissance de mouvement (wiimote).


\begin{DoxyItemize}
\item Product Owner \+: D\+E\+S\+M\+E\+U\+L\+L\+E\+S Gireg 
\item Client \+: Association “\+Vivre le Monde” 
\item Scrum Team \+: 
\begin{DoxyItemize}
\item B\+I\+G\+E\+A\+R\+D Charles-\/\+Henry 
\item L\+E\+J\+E\+U\+N\+E Pierre 
\item PÉ\+R\+I\+CÉ Robin
\end{DoxyItemize}
\end{DoxyItemize}\hypertarget{index_install_sec}{}\section{Utilisation}\label{index_install_sec}
\hypertarget{index_step1}{}\subsection{1 -\/ L\textquotesingle{}interface}\label{index_step1}
Lancez l\textquotesingle{}application à partir de l\textquotesingle{}exécutable. Au lancement de l\textquotesingle{}application un métronome apparait, il est constitué d\textquotesingle{}un cylindre dont la taille est fonction de la durée de la boucle ainsi que de sphères représentants les sons de la boucle.

\hypertarget{index_step2}{}\subsection{2 -\/ Intéragir avec l\textquotesingle{}application}\label{index_step2}
Pour intéragir avec l\textquotesingle{}application il existe plusieurs méthodes (le clavier, les boutons dans l\textquotesingle{}ihm et la wiimote).

\begin{TabularC}{4}
\hline
{\bfseries Intéraction} &{\bfseries Touche du clavier} &{\bfseries Bouton interface} &{\bfseries Bouton wiimote}  \\\cline{1-4}
Augmenter le nombre de B\+P\+M de la prochaine boucle créée &\begin{center}t\end{center}  &\begin{center}undefined\end{center}  &\begin{center}undefined\end{center}   \\\cline{1-4}
Réduire le nombre de B\+P\+M de la prochaine boucle créée &\begin{center}g\end{center}  &\begin{center}undefined\end{center}  &\begin{center}undefined\end{center}   \\\cline{1-4}
Augmenter la mesure de la prochaine boucle créée &\begin{center}y\end{center}  &\begin{center}undefined\end{center}  &\begin{center}undefined\end{center}   \\\cline{1-4}
Réduire la mesure de la prochaine boucle créée &\begin{center}h\end{center}  &\begin{center}undefined\end{center}  &\begin{center}undefined\end{center}   \\\cline{1-4}
Ajouter une boucle &\begin{center}a\end{center}  &\begin{center}undefined\end{center}  &\begin{center}undefined\end{center}   \\\cline{1-4}
Se déplacer vers la boucle de gauche &\begin{center}flèche gauche\end{center}  &\begin{center}undefined\end{center}  &\begin{center}flèche gauche\end{center}   \\\cline{1-4}
Se déplacer vers la boucle de droite &\begin{center}flèche droite\end{center}  &\begin{center}undefined\end{center}  &\begin{center}flèche droite\end{center}   \\\cline{1-4}
Ajouter un son de basse &\begin{center}q\end{center}  &\begin{center}undefined\end{center}  &\begin{center}A + mouvement wiimote\end{center}   \\\cline{1-4}
Ajouter un son de clave &\begin{center}s\end{center}  &\begin{center}undefined\end{center}  &\begin{center}B + mouvement wiimote\end{center}   \\\cline{1-4}
Ajouter un son de bip &\begin{center}d\end{center}  &\begin{center}undefined\end{center}  &\begin{center}A + B + mouvement wiimote\end{center}   \\\cline{1-4}
Mute/\+De-\/mute le métronome &\begin{center}m\end{center}  &\begin{center}undefined\end{center}  &\begin{center}undefined\end{center}    \\\cline{1-4}
Supprimer une boucle &\begin{center}x\end{center}  &\begin{center}undefined\end{center}  &\begin{center}undefined\end{center}   \\\cline{1-4}
\end{TabularC}
